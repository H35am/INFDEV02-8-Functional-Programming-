\section{General description}

Functional programming and functional programming languages are increasing in popularity for multiple reasons and in multiple ways, to the point that even mainstream languages such as Python, C++, C\#, and Java are being extended with more and more functional programming features such as tuples, lambda's, higher order functions, and even monads such as LINQ and async/await. Whole architectures such as the popular map/reduce are strongly inspired by functional programming.

``Java™ developers should learn functional paradigms now, even if they have no immediate plans to move to a functional language such as Scala or Clojure. Over time, all mainstream languages will become more functional'' [IBM].

``LISP is worth learning for a different reason — the profound enlightenment experience you will have when you finally get it. That experience will make you a better programmer for the rest of your days, even if you never actually use LISP itself a lot.'' – Eric S. Raymond

``SQL, Lisp, and Haskell are the only programming languages that I've seen where one spends more time thinking than typing.'' – Philip Greenspun

``I do not know if learning Haskell will get you a job. I know it will make you a better software developer.'' – Larry O’ Brien

The reason for this growth is to be found in the safe and deep expressive power of functional languages, which are capable of recombining simpler elements into powerful, complex other elements with less space for mistakes and more control in the hands of the programmer. This comes at a fundamental cost: functional languages are structurally different from imperative and object oriented languages, and thus a new mindset is required of the programmer that wishes to enter this new world. Moreover, functional languages often require more thought and planning, and are thus experienced, especially by beginners, as somewhat less flexible and supporting of experimentation.

\subsection{Relationship with other didactic units}
This module completes and perfects the understanding and knowledge of programming that was set up in the preceding INFDEV courses.
