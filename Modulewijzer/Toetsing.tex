\section{Assessment}
The course is tested with two exams:
A series of assignments which have to be handed in, but will not be graded offline. There will be an oral/practicum check, which is based on the assignments,
and a written exam. The final grade is determined as follows: \\

\texttt{if exam grade $ >= 5.0 $ then practicum-grade else 0}

\paragraph*{Motivation for grade}
A professional software developer is required to be able to program code which is, at the very least, \textit{correct}.

In order to produce correct code, we expect students to show:
\begin{inparaenum}[\itshape i\upshape)]
\item a foundation of knowledge about how the semantics of the programming language actually work;
\item fluency when actually writing the code.
\end{inparaenum}

The quality of the programmer is ultimately determined by his actual code-writing skills, therefore the written exam will contain require you to write code. This ensures that each student is able to show that his work is his own and that he has adequate understanding of its mechanisms.



\subsection{Theoretical examination \modulecode}
The general shape of an exam for \texttt{\modulecode} is made up of a short series of highly structured open questions.
In each exam the content of the questions will change, but the structure of the questions will remain the same.
For the structure (and an example) of the theoretical exam, see the appendix.


\subsection{Practical examination \modulecode}
There is only one assignment, which is mandatory, and formatively assessed for feedback.

\begin{itemize}
  \item All assignments are to be uploaded to N@tschool or Classroom in the required space (Inlevermap or assignment);
  \item Each assignment is designed to assess the students knowledge related to one or more learning goal.
          If the teacher is unable te assess the students' ability related to the appropriate learning goal based on his work, then no points will be awarded for that part.
  \item \textit{The teachers still reserves the right to check the practicums handed in by each student, and to use it for further evaluation.}
  \item The university rules on fraude and plagiarism (Hogeschoolgids art. 11.10 -- 11.12) also apply to code;
  
The practical exam requires to complete one of the following assignments.

\begin{itemize}
	\item A 2D simulation of a supermarket with customers, cash registers, and various aisles
	\item A 2D simulation of a supply chain with trucks, containers, and ships
	\item An interpreter for a Python-like language (with a parser for an extra challenge)
	\item An interpreter for the lambda calculus  (with a parser for an extra challenge)	
\end{itemize}

The solution should include the following features from functional programming languages: (\textit{i}) Tuples, (\textit{ii}) Discriminated Unions, 
(\textit{iii}) Records, (\textit{iv}) Functions, and (\textit{v}) Recursive functions. Each of these features will be removed during the oral check and the student 
will be asked to rewrite those parts. Each correct reproduced feature gives the student 2 points. The student can choose the details of the simulation or interpreter himself.

\end{itemize}
