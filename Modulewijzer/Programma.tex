\section{Course program}
The course is structured into eight lectures.
The eight lectures take place during the eight weeks of the course, but are not necessarily in a one-to-one correspondance with the course weeks.

\subsection{Chapter 1 - foundations (weeks 1, and 2)}
\paragraph*{Topics}
\begin{itemize}
	\item Recap of imperative programming language semantics: the \textit{shared memory} model;
	\item Functional programming concepts: the lambda calculus and beta reduction;
	\item Adding static typing: the simply typed lambda calculus;
	\item Making the language usable: delta rules.
\end{itemize}

\subsection{Chapter 2 - practical applications (weeks 3, 4, 5)}
\paragraph*{Topics}			
\begin{itemize}
	\item From the lambda calculus to F\# (with practical lecture);
	\item Lazy evaluation;
	\item From the lambda calculus to Haskell (with practical lecture);
	\item Advanced constructs: lists (and list comprehensions), records, tuples, discriminated unions;
	\item Interop with other languages.
\end{itemize}

\subsection{Chapter 3 - patterns and practice (weeks 6, 7, 8)}
\paragraph*{Topics}			
\begin{itemize}
	\item Traversable entities that self-update;
	\item Monads introduction;
	\item Traversable entities that self-update with coroutines (with practical lecture).
\end{itemize}
