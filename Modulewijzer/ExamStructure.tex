\section*{Structure of exam \modulecode}
The general shape of a theoretical exam for \texttt{DEV 3} is made up of only two, highly structured open questions.

\paragraph{Question 1: } \ \\

\textbf{General shape of the question:} \textit{Given the following lambda program, and a series of delta rules, show the beta reductions for this program.}

\textbf{Concrete example of question:}

\textbf{Program:}
\lstset{numbers=left,basicstyle=\ttfamily\small}\lstset{language=[Sharp]C}\begin{lstlisting}
(((TRUE $\wedge$ TRUE) T) F)
\end{lstlisting}

\textbf{Delta rules:}
\lstset{numbers=left,basicstyle=\ttfamily\small}\lstset{language=[Sharp]C}\begin{lstlisting}
TRUE $\equiv$ $\lambda$t$\rightarrow$f$\rightarrow$t
FALSE $\equiv$ $\lambda$t$\rightarrow$f$\rightarrow$f
$\wedge$ $\equiv$ $\lambda$p$\rightarrow$q$\rightarrow$((p q) p)
\end{lstlisting}

\textbf{Concrete example of answer:}

\lstset{basicstyle=\ttfamily\small}\lstset{numbers=none}\lstset{language=ML}\begin{lstlisting}
(((TRUE $\wedge$ TRUE) T) F)

(((( (*@\underline{$\wedge$}@*) TRUE) TRUE) T) F)

((((($\lambda$p$\rightarrow$q$\rightarrow$((p q) p)) TRUE) TRUE) T) F)

((((($\lambda$p$\rightarrow$q$\rightarrow$((p q) p)) (*@\underline{TRUE}@*)) TRUE) T) F)

((((($\lambda$p$\rightarrow$q$\rightarrow$((p q) p)) ($\lambda$t$\rightarrow$f$\rightarrow$t)) TRUE) T) F)

((((($\lambda$p$\rightarrow$q$\rightarrow$((p q) p)) ($\lambda$t$\rightarrow$f$\rightarrow$t)) (*@\underline{TRUE}@*)) T) F)

((((($\lambda$p$\rightarrow$q$\rightarrow$((p q) p)) ($\lambda$t$\rightarrow$f$\rightarrow$t)) ($\lambda$t$\rightarrow$f$\rightarrow$t)) T) F)

((((*@\underline{(($\lambda$p$\rightarrow$q$\rightarrow$((p q) p)) ($\lambda$t$\rightarrow$f$\rightarrow$t))}@*) ($\lambda$t$\rightarrow$f$\rightarrow$t)) T) F)

(((($\lambda$q$\rightarrow$((($\lambda$t$\rightarrow$f$\rightarrow$t) q) ($\lambda$t$\rightarrow$f$\rightarrow$t))) ($\lambda$t$\rightarrow$f$\rightarrow$t)) T) F)

(((*@\underline{(($\lambda$q$\rightarrow$((($\lambda$t$\rightarrow$f$\rightarrow$t) q) ($\lambda$t$\rightarrow$f$\rightarrow$t))) ($\lambda$t$\rightarrow$f$\rightarrow$t))}@*) T) F)

((((($\lambda$t$\rightarrow$f$\rightarrow$t) ($\lambda$t$\rightarrow$f$\rightarrow$t)) ($\lambda$t$\rightarrow$f$\rightarrow$t)) T) F)

((((*@\underline{(($\lambda$t$\rightarrow$f$\rightarrow$t) ($\lambda$t$\rightarrow$f$\rightarrow$t))}@*) ($\lambda$t$\rightarrow$f$\rightarrow$t)) T) F)

(((($\lambda$f$\rightarrow$t$\rightarrow$f$\rightarrow$t) ($\lambda$t$\rightarrow$f$\rightarrow$t)) T) F)

(((*@\underline{(($\lambda$f$\rightarrow$t$\rightarrow$f$\rightarrow$t) ($\lambda$t$\rightarrow$f$\rightarrow$t))}@*) T) F)

((($\lambda$t$\rightarrow$f$\rightarrow$t) T) F)

((($\lambda$t$\rightarrow$f$\rightarrow$t) T) F)

((*@\underline{(($\lambda$t$\rightarrow$f$\rightarrow$t) T)}@*) F)

(($\lambda$f$\rightarrow$T) F)

(*@\underline{(($\lambda$f$\rightarrow$T) F)}@*)

T
\end{lstlisting}
	

\textbf{Points:} \textit{4 (50\% of total).}

\textbf{Grading:} \textit{Full points for more all correct steps and result. Half points if correct result is found with some reduction mistakes. Zero points otherwise.}

\textbf{Associated learning objective:} \glsfirst{red}

\ \\ 

\paragraph{Question 2: } \ \\

\textbf{General shape of question:} \textit{Given the following lambda calculus program, and a series of typing rules, give the full typing derivation for the program.}

\begin{comment}
\textbf{Concrete example of question:} 

\lstset{numbers=left,basicstyle=\ttfamily\small}\lstset{language=[Sharp]C}


\textbf{Concrete example of answer:} \textit{}

\begin{lstlisting}
TODO
\end{lstlisting}
\end{comment}

\textbf{Points:} \textit{4 (50\% of total).}

\textbf{Grading:} \textit{Full points for fully correct type derivation. Half points for minor mistakes but correct overall structure. Zero points otherwise.}

\textbf{Associated learning objective:} \glsfirst{typ}

\ \\
